\documentclass{article}

\title{Metashift}
\date{\today}
\author{Matthew D'Souza \& Dongyu Zheng}

\begin{document}
\maketitle
\pagenumbering{gobble}

\newpage
\tableofcontents

\newpage
\pagenumbering{arabic}


%%%%%%%%%%%%%%%%%%%%%%%%%%%%%%
%     BEGIN DOCUMENTATION    %
%%%%%%%%%%%%%%%%%%%%%%%%%%%%%%

\section{Abstract}

Created for the League of Legends API Challenge 2.0, Metashift is a web app which uses Riot Games\'API to query the game information and present a graphical analysis of thousands of League of Legends matches before and after the 5.13 patch. Many of the changes were made to the items with the Ability Power stat. The goal of this project is to systematically look for a shift in the \"metagame\", the most popular (and often most successful) characters played both before and after patch 5.13.


\section{Features}

\subsection{Our Ideas}
\begin{itemize}
    \item pick/win/ban rates of champions, with graphing and charting
    \item purchase/win rates of items with respect to champions, with graphing and charting
    \item time of purchases (earlier/later \"power spikes\")
    \item look for any emerging meta picks
    \item sortable by region, queue, rank if applicable
    \item more general game trends of before \& after: first dragon, first baron, first tower, average kills at a certain time, etc.
    \item before and after champion roles calculated with K-Means Clustering
\end{itemize}

\subsection{Selected for Development}
\begin{itemize}
    \item pick/win/ban rates of champions, with graphing and charting
    \item purchase/win rates of items, with graphing and charting
    \item before and after champion roles calculated with K-Means Clustering
\end{itemize}


\section{Backend}

\subsection{Database}
lorem ipsum

\subsection{Downloading}
lorem ipsum

\subsection{Counting}
lorem ipsum

\subsection{K-Means Clustering}
lorem ipsum

\subsection{Exporting}
lorem ipsum



\section{Frontend}

\subsection{Graphing}
lorem ipsum

\subsection{Charting}
lorem ipsum


%%%%%%%%%%%%%%%%%%%%%%%%%%%%%%
%      END DOCUMENTATION     %
%%%%%%%%%%%%%%%%%%%%%%%%%%%%%%

\end{document}